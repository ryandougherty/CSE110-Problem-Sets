\documentclass{article}

\usepackage [margin=1in]{geometry}
\usepackage[usenames,dvipsnames]{color}
\usepackage{listings}
\lstset{language=Java,
	basicstyle=\ttfamily,
	keywordstyle=\color{blue}\ttfamily,
	stringstyle=\color{red}\ttfamily,
	showstringspaces=false,
	commentstyle=\color{ForestGreen}\ttfamily,
	morecomment=[l][\color{magenta}]{\#}}
\usepackage [autostyle]{csquotes}
\usepackage [english]{babel}
\usepackage {verbatim}

\title{CSE110 Review Questions (Solutions)}
\author{Prepared by Ryan Dougherty}
\date{}

\begin{document}
\maketitle

\section{Static Variables and Methods}

\newcounter{question_num}

\setcounter{question_num}{1}
\paragraph{Question \arabic{question_num}}
What is a static variable? What is a static method?
\newline {\color{red}Answer: A static variable is shared among all instances of a class. A static method is invoked using the class name.}

\addtocounter{question_num}{1}
\paragraph{Question \arabic{question_num}}
Using the code below, how many copies of the variable number exist after instantiating 374 different AmazingClass objects?
\begin{lstlisting}
public class AmazingClass {
	private static int number;

	public AmazingClass(int a) {
		number = a;
	}

	public int twice() {
		number *= 2;
		return number;
	}
}
\end{lstlisting}
{\color{red}Answer: 1}

\addtocounter{question_num}{1}
\paragraph{Question \arabic{question_num}}
Using the code from above, what is the value of number after each of the following statements? (For each part, assume the preceding parts have already been executed).
\begin{lstlisting}
AmazingClass ac1 = new AmazingClass(3); 
AmazingClass ac2 = new AmazingClass(7);
ac1.twice();
ac2.twice();
\end{lstlisting}
{\color{red}Answers:
\newline 3
\newline 7
\newline 14
\newline 28
}

\end{document}