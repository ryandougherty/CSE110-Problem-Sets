\documentclass{article}

\usepackage [margin=1in]{geometry}
\usepackage[usenames,dvipsnames]{color}
\usepackage{listings}
\lstset{language=Java,
	basicstyle=\ttfamily,
	keywordstyle=\color{blue}\ttfamily,
	stringstyle=\color{red}\ttfamily,
	showstringspaces=false,
	commentstyle=\color{ForestGreen}\ttfamily,
	morecomment=[l][\color{magenta}]{\#}}
\usepackage [autostyle]{csquotes}
\usepackage [english]{babel}
\usepackage {verbatim}

\title{CSE110 Review Questions (Solutions)}
\author{Prepared by Ryan Dougherty}
\date{}

\begin{document}
\maketitle

\section{Methods}

\newcounter{question_num}

\setcounter{question_num}{1}
\paragraph{Question \arabic{question_num}}
Write a boolean method called allDifferent that takes 3 int numbers and returns true if the numbers are all different and false otherwise.
\newline {\color{red} Answer:}
\begin{lstlisting}
public boolean allDifferent(int num1, int num2, int num3) {
	if (num1 != num2 && num1 != num3 && num2 != num3) {
		return true;
	} else {
		return false;
	}
}
\end{lstlisting}

\addtocounter{question_num}{1}
\paragraph{Question \arabic{question_num}}
Write a boolean method called isPrime that takes in an int number, and returns true if the number is prime, and false otherwise.
\newline {\color{red}Answer:}
\begin{lstlisting}
public boolean isPrime(int n) {
	// Question: how can we improve the performance of this loop? 
	// (Hint: what is the max number relative to n that can divide into n?)
	for (int i = 2; i < n; i++) {
		if (n % i == 0) {
			return false;
		}
	}
	return true;
}
\end{lstlisting}

\end{document}