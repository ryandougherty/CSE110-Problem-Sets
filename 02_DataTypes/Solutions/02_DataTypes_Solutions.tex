\documentclass{article}

\usepackage [margin=1in]{geometry}
\usepackage[usenames,dvipsnames]{color}
\usepackage{listings}
\lstset{language=Java,
	basicstyle=\ttfamily,
	keywordstyle=\color{blue}\ttfamily,
	stringstyle=\color{red}\ttfamily,
	showstringspaces=false,
	commentstyle=\color{ForestGreen}\ttfamily,
	morecomment=[l][\color{magenta}]{\#}}
\usepackage [autostyle]{csquotes}
\usepackage [english]{babel}
\usepackage {verbatim}


\date{}

\begin{document}
\title{\textbf{CSE110 Review Questions \\
Prepared by Ryan Dougherty}}
\maketitle

\newcounter{question_num}
\setcounter{question_num}{1}

\section*{Data Types}

\setcounter{question_num}{1}
\paragraph{Question \arabic{question_num}}
Give the output of the following program:
\begin{lstlisting}
public class Example {
	public static void main(String[] args) {
		int y = 2, z = 1;
		z = y * 2;
		System.out.print(y + z);
	}
}
\end{lstlisting}
{\color{red}Answer: 6. This is because y is 2, and 2*2 is stored in z. Now, z=4, and y is the same value (2). Therefore, 4+2 = 6.}

\addtocounter{question_num}{1}
\paragraph{Question \arabic{question_num}}
Consider the following program:
\begin{lstlisting}
public class Example {
	public static void main(String[] args) {
		String str = new String("Arizona state university");
		char ch1 = str.toLowerCase().toUpperCase().charAt(0);
		char ch2 = str.toUpperCase().charAt(8);
		char ch3 = str.toUpperCase().charAt(str.length() - 1);
		System.out.println("character 1 is: " + ch1);
		System.out.println("character 2 is: " + ch2);
		System.out.println("character 3 is: " + ch3);
	}
}
\end{lstlisting}
What will be the output?
\newline
{\color{red}character 1 is: A
\newline character 2 is: S
\newline character 3 is: Y}

\addtocounter{question_num}{1}
\paragraph{Question \arabic{question_num}}
Consider the following program:
\begin{lstlisting}
public class Example {
	public static void main(String[] args) {
		int num1 = 4, num2 = 5;
		System.out.println("4" + "5");
		System.out.println(num1 + num2);
		System.out.println("num1" + "num2");
		System.out.println(4+5);
	}
}
\end{lstlisting}
What will be the output?
\newline
{\color{red}Answer:
\newline 45
\newline 9
\newline num1num2
\newline 9}

\addtocounter{question_num}{1}
\paragraph{Question \arabic{question_num}}
Which of the following invokes the method length() of the object str and stores the result in val of type int?
\begin{lstlisting}
a) int val = str.length();
b) int val = length.str();
c) int val = length().str;
d) int val = length(str);
\end{lstlisting}
{\color{red}Answer: A}

\addtocounter{question_num}{1}
\paragraph{Question \arabic{question_num}}
Evaluate each of the following expressions.
\begin{lstlisting}
String s = "Programming is Fun";
String t = "Workshop is cool";
a) System.out.println(s.charAt(0) + t.substring(3, 4));
b) System.out.println(t.substring(7));
\end{lstlisting}
{\color{red}Answers:
\newline a) Pk
\newline b) p is cool}

\addtocounter{question_num}{1}
\paragraph{Question \arabic{question_num}}
Evaluate each of the following expressions.
\begin{lstlisting}
int j = 11;
int k = 3;
String s = "Ford Rivers";
a) j / k
b) j % k
c) s.substring(1, 5)
d) s.length()
e) s.charAt(3)
\end{lstlisting}
{\color{red}Answers:
\newline a) 3
\newline b) 2
\newline c) ord
\newline d) 11
\newline e) d
}

\addtocounter{question_num}{1}
\paragraph{Question \arabic{question_num}}
True or False? The type char is a primitive data type.
{\color{red}Answer: True}

\addtocounter{question_num}{1}
\paragraph{Question \arabic{question_num}}
True or False? The type String is a primitive data type.
{\color{red}Answer: False}

\addtocounter{question_num}{1}
\paragraph{Question \arabic{question_num}}
Write a Java program that asks the user for the radius of a circle and finds the area of the circle.
\newline {\color{red}Answer:}
\begin{lstlisting}
import java.util.Scanner;
public class Circle {
	public static void main(String[] args) {
		final double PI = 3.14159;
		Scanner scan = new Scanner(System.in);
		System.out.println("Enter radius: ");
		double radius = scan.nextDouble();
		double area = PI * radius * radius;
		System.out.println("The area is: " + area);
	}
}
\end{lstlisting}

\addtocounter{question_num}{1}
\paragraph{Question \arabic{question_num}}
Write a Java program that prompts the user to enter 2 integers. Print the smaller of the 2 integers.
\newline {\color{red}Answer:}
\begin{lstlisting}
import java.util.Scanner;
public class Smaller {
	public static void main(String[] args) {
		Scanner scan = new Scanner(System.in);
		System.out.println("Enter 2 integers:");
		int a = scan.nextInt();
		int b = scan.nextInt();
		System.out.println("The smaller integer is " + Math.min(a, b));
	}
}
\end{lstlisting}

\addtocounter{question_num}{1}
\paragraph{Question \arabic{question_num}}
Write the output of the following program:
\begin{lstlisting}
public class Question {
	public static void main(String[] args) {
		String str = "hello";
		System.out.println("abcdef".substring(1, 3));
		System.out.println("pizza".length());
		System.out.println(str.replace('h', 'm'));
		System.out.println("hamburger".substring(0, 3));
		System.out.println(str.charAt(1));
		System.out.println(str.equals("hello"));
		System.out.println("pizza".toUpperCase());
		System.out.println(Math.pow(2, 4));
		double num4 = Math.sqrt(16);
		System.out.println(num4);
	}
}
\end{lstlisting}
{\color{red}Answers:
\newline bc
\newline 5
\newline mello
\newline ham
\newline e
\newline true
\newline PIZZA
\newline 16.0
\newline 4.0
}


\addtocounter{question_num}{1}
\paragraph{Question \arabic{question_num}}
Write the output of the following program:
\begin{lstlisting}
public class Question {
	public static void main(String[] args) {
		String s1 = new String("Clinton, Hillary");
		String s2 = new String("Obama, Barack");
		System.out.println(s1.charAt(2));
		System.out.println(s1.charAt(s1.length() - 1));
		System.out.println(s2.toUpperCase());
		System.out.println(s2.substring(s2.indexOf(",") + 2, s2.length());
	}
}
\end{lstlisting}
{\color{red}Answers:
\newline i
\newline y
\newline OBAMA, BARACK
\newline Barack
}

\addtocounter{question_num}{1}
\paragraph{Question \arabic{question_num}}
What value is contained in the integer variable length after the following statements are executed?
\begin{lstlisting}
length = 5;
length += 3;
length = length * 2;
\end{lstlisting}
{\color{red}Answer: 16}

\addtocounter{question_num}{1}
\paragraph{Question \arabic{question_num}}
What is the result of 2/4 when evaluated in Java? Why?
{\color{red}Answer: 0. It is an integer division.}

\end{document}