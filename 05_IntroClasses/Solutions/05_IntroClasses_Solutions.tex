\documentclass{article}

\usepackage [margin=1in]{geometry}
\usepackage[usenames,dvipsnames]{color}
\usepackage{listings}
\lstset{language=Java,
	basicstyle=\ttfamily,
	keywordstyle=\color{blue}\ttfamily,
	stringstyle=\color{red}\ttfamily,
	showstringspaces=false,
	commentstyle=\color{ForestGreen}\ttfamily,
	morecomment=[l][\color{magenta}]{\#}}
\usepackage [autostyle]{csquotes}
\usepackage [english]{babel}
\usepackage {verbatim}

\title{CSE110 Review Questions (Solutions)}
\author{Prepared by Ryan Dougherty}
\date{}

\begin{document}
\maketitle

\section{Introduction to Classes}

\newcounter{question_num}

\setcounter{question_num}{1}
\paragraph{Question \arabic{question_num}}
Which of the following enforces Encapsulation?
\newline a) Make instance variables private
\newline b) Make methods public
\newline c) Make the class final
\newline d) Both a and b
\newline e) All of the above
\newline {\color{red}Answer: D}

\addtocounter{question_num}{1}
\paragraph{Question \arabic{question_num}}
Use the following class to answer the questions below:
\begin{lstlisting}
public class Store {
	private int quantity;
	private double price;

	public Store(int q, double p) {
		quantity = q;
		price = p;
	}

	public int getQuantity() {
		return quantity;
	}

	public void setPrice(double p) {
		price = p;
	}

	public double calcTotal() {
		return price * quantity;
	}
}
\end{lstlisting}
a) What is the name of the class? {\color{red}Store}
\newline b) List all instance variables of the class. {\color{red}quantity, price}
\newline c) List all methods of the class. {\color{red}Store(int, double), getQuantity(), setPrice(double), calcTotal()}
\newline d) List all mutators in the class. {\color{red}setPrice(double)}
\newline e) List all accessors in the class. {\color{red}getQuantity()}
\newline f) List which method is the constructor. {\color{red}Store(int, double)}
\newline g) Write a mutator for the quantity.
\newline{\color{red}Answer:}
\begin{lstlisting}
public void setQuantity(int q) {
	quantity = q;
}
\end{lstlisting}

h) Write an accessor for the price.
\newline{\color{red}Answer:}
\begin{lstlisting}
public double getPrice() {
	return price;
}
\end{lstlisting}

i) Write a line of code that will create an instance called videoStore that has quantity 100 and a price of 5.99.
\newline{\color{red}Answer:}
\begin{lstlisting}
Store videoStore = new Store(100, 5.99);
\end{lstlisting}

j) Call the calcTotal method with the videoStore object (from part i) to print out the total.
\newline{\color{red}Answer:}
\begin{lstlisting}
System.out.println("Total: " + videoStore.calcTotal());
\end{lstlisting}

\addtocounter{question_num}{1}
\paragraph{Question \arabic{question_num}}
True or False? If no constructor is provided, then Java automatically provides a default constructor. {\color{red} Answer: True. Java will automatically provide a default constructor if none is given. However, if any other constructor is given, the default constructor can no longer be used.}

\addtocounter{question_num}{1}
\paragraph{Question \arabic{question_num}}
True or False? A method must have at least 1 return statement. {\color{red} Answer: False. Any method with return type void is not required to have a return statement. However, if it does have a return statement, it must not have a value associated with it (i.e. "return 5;" is not allowed for void methods, but "return;" is.).}

\addtocounter{question_num}{1}
\paragraph{Question \arabic{question_num}}
Correct the following class definition if you think it will not work:
\begin{lstlisting}
public class Student {
	private String name, major;

	public Student() {
		name = "???";
		major = "xxx";
	}

	public Student(String n, String m) {
		n = name;
		m = major;
	}

	public String getMajor() {
		return m;
	}

	public String getName() {
		return n;
	}
}
\end{lstlisting}
{\color{red}Answer: There are problems in the assignment in the constructor, "n = name" and "m = major" should be the other way around. Also, the "return m" in getMajor and "return n" in getName need to be "return major" and "return name", respectively.}

\addtocounter{question_num}{1}
\paragraph{Question \arabic{question_num}}
Implement a class called AsuStudent. The class should keep track of the student's name, number of classes registered, hours spent per week for a class (consider a student devotes the same amount of time for each of his/her classes per week). Implement a toString method to show the name and number of classes registered by a student, a getName method to return the name of the student, a getTotalHours method to return the total number of hours per week, and a setHours method to set the number of hours the student devotes for each class.
 {\color{red} \newline Answer:}
\begin{lstlisting}
public class AsuStudent {
	private String sName;
	private int classNum, hrPerWeek;
	
	public AsuStudent(String name, int class, int hr) {
		sName = name;
		classNum = class;
		hrPerWeek = hr;
	}

	public String toString() {
		return sName + " " + classNum + " " + hrPerWeek;
	}

	public String getName() {
		return sName;
	}

	public int getTotalHours() {
		return classNum * hrPerWeek;
	}

	public void setHours(int time) {
		hrPerWeek = time;
	}
}
\end{lstlisting}

\end{document}