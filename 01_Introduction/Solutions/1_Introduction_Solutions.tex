\documentclass{article}

\usepackage [margin=1in]{geometry}
\usepackage[usenames,dvipsnames]{color}
\usepackage{listings}
\lstset{language=Java,
	basicstyle=\ttfamily,
	keywordstyle=\color{blue}\ttfamily,
	stringstyle=\color{red}\ttfamily,
	showstringspaces=false,
	commentstyle=\color{ForestGreen}\ttfamily,
	morecomment=[l][\color{magenta}]{\#}}
\usepackage [autostyle]{csquotes}
\usepackage [english]{babel}
\usepackage {verbatim}

\title{CSE110 Review Questions (Solutions)}
\author{Prepared by Ryan Dougherty}
\date{}

\begin{document}
\maketitle

\section{Introduction}

\newcounter{question_num}
\setcounter{question_num}{1}
\paragraph{Question \arabic{question_num}}
What does a compiler do?
\newline a) Translates machine instructions to higher level languages
\newline b) Translates programs written in a high-level language into machine code
\newline c) Translates user programs to Java Programs
\newline d) None of the above
\newline {\color{red}Answer: C}

\addtocounter{question_num}{1}
\paragraph{Question \arabic{question_num}}
Consider the following Java Program:
\begin{lstlisting}
public class VendingMachine {
	public static void main(String[] args) {
		System.out.println("Please insert 25c");
	}
}
\end{lstlisting}
By what name would you save this program on your hard disk?
\newline {\color{red}Answer: VendingMachine.java}

\addtocounter{question_num}{1}
\paragraph{Question \arabic{question_num}}
Is Java a functional language, procedural language, object-oriented language, or logic language?
\newline {\color{red}Answer: Java is an object-oriented language.}

\end{document}