\documentclass{article}

\usepackage [margin=1in]{geometry}
\usepackage[usenames,dvipsnames]{color}
\usepackage{listings}
\lstset{language=Java,
	basicstyle=\ttfamily,
	keywordstyle=\color{blue}\ttfamily,
	stringstyle=\color{red}\ttfamily,
	showstringspaces=false,
	commentstyle=\color{ForestGreen}\ttfamily,
	breaklines=true,
	morecomment=[l][\color{magenta}]{\#}}
\usepackage [autostyle]{csquotes}
\usepackage [english]{babel}
\usepackage {verbatim}

\title{CSE110 Review Questions (Solutions)}
\author{Prepared by Ryan Dougherty}
\date{}

\begin{document}
\maketitle

\section{Decision - Control Structures}

\newcounter{question_num}

\setcounter{question_num}{1}
\paragraph{Question \arabic{question_num}}
What is the output of the following code?
\begin{lstlisting}
int depth = 8;
if (depth >= 8) {
	System.out.print("Danger: ");
	System.out.print("deep water. ");
}
System.out.println("No swimming allowed.");
\end{lstlisting}
{\color{red}Answer: Danger: deep water. No swimming allowed.}

\addtocounter{question_num}{1}
\paragraph{Question \arabic{question_num}}
What is the output of the following code?
\begin{lstlisting}
int depth = 12;
int temp = 42;
System.out.print("The water is: ");
if (depth >= 8)
	System.out.print("deep ");
if (temp <= 50 && depth <= 12)
	System.out.print("cold ");
System.out.println(" wet.");
\end{lstlisting}
{\color{red}Answer: The water is deep cold wet.}

\addtocounter{question_num}{1}
\paragraph{Question \arabic{question_num}}
Consider the following code:
\begin{lstlisting}
String str1 = "Java is fun";
String str2 = "Java is fun";
if ( /* */ )
	System.out.println("String1 and String2 are the same");
else
	System.out.println("String1 and String2 are different");
\end{lstlisting}
Fill in the missing condition to check if str1 and str2 are the same.
\newline {\color{red}Answer: str1.equals(str2)}

\addtocounter{question_num}{1}
\paragraph{Question \arabic{question_num}}
If k holds a value of the type int, then the value of the expression:
\begin{lstlisting}
k <= 10 || k > 10
\end{lstlisting}
a) must be true
\newline b) must be false
\newline c) could be either true or false
\newline d) is a value of type int
\newline {\color{red}Answer: A}

\addtocounter{question_num}{1}
\paragraph{Question \arabic{question_num}}
Write a program that asks for 3 integers and prints the median value of the three integers.
\newline {\color{red}Answer:}
\begin{lstlisting}
import java.util.Scanner;
public class ThreeIntegers {
	public static void main(String[] args) {
		Scanner scan = new Scanner(System.in);
		int num1 = scan.nextInt();
		int num2 = scan.nextInt();
		int num3 = scan.nextInt();
		int result = 0;
		if (num1 > num2 && num1 < num3)
			result = num1;
		else if (num1 < num2 && num1 > num3)
			result = num1;
		else if (num2 > num1 && num2 < num3)
			result = num2;
		else if (num2 < num1 && num2 > num3)
			result = num2;
		else
			result = num3;

		System.out.println("The median of " + num1 + ", " + num2 + ", and " + num3 + " is " + result);
	}
}
\end{lstlisting}

\addtocounter{question_num}{1}
\paragraph{Question \arabic{question_num}}
Evaluate the following expressions, assuming that x = -2 and y = 3.
\begin{lstlisting} 
a) x <= y 
b) (x < 0) || (y < 0)
c) (x <= y) && (x < 0)
d) ((x + y) > 0) && !(y > 0)
\end{lstlisting}
{\color{red}Answers:
\newline a) true
\newline b) true
\newline c) true
\newline d) false}

\addtocounter{question_num}{1}
\paragraph{Question \arabic{question_num}}
Write the output of the following code:
\begin{lstlisting}
int grade = 45;
if (grade >= 70)
	System.out.println("passing");
if (grade < 70)
	System.out.println("dubious");
if (grade < 60)
	System.out.println("failing");
\end{lstlisting}
{\color{red}Answer:
\newline dubious
\newline failing
}

\addtocounter{question_num}{1}
\paragraph{Question \arabic{question_num}}
Write the output of the following code:
\begin{lstlisting}
String option = "A";
if (option.equals("A"))
	System.out.println("addRecord");
if (option.compareTo("A") == 0)
	System.out.println("deleteRecord");
\end{lstlisting}
{\color{red}Answer:
\newline addRecord
\newline deleteRecord
}

\addtocounter{question_num}{1}
\paragraph{Question \arabic{question_num}}
Write the output of the following code:
\begin{lstlisting}
double x = -1.5;
if (x < -1.0)
	System.out.println("true");
else
	System.out.println("false");
	System.out.println("after if...else");
\end{lstlisting}
{\color{red}Answer:
\newline true
\newline after if...else
}

\addtocounter{question_num}{1}
\paragraph{Question \arabic{question_num}}
Write the output of the following code:
\begin{lstlisting}
int j = 8;
double x = -1.5;
if (x >= j)
	System.out.println("x is high");
else
	System.out.println("x is low");
\end{lstlisting}
{\color{red}Answer:
\newline x is low
}

\addtocounter{question_num}{1}
\paragraph{Question \arabic{question_num}}
Write the output of the following code:
\begin{lstlisting}
double x = -1.5;
if (x <= 0.0) {
	if (x < 0.0)
		System.out.println("neg");
	else
		System.out.println("zero");
}
else
	System.out.println("pos");
\end{lstlisting}
{\color{red}Answer:
\newline neg
}

\addtocounter{question_num}{1}
\paragraph{Question \arabic{question_num}}
Write code that ensures that an int variable called number is an odd integer.
{\color{red}Answer:}
\begin{lstlisting}
if ((number % 2) == 0)
	number++;
// number at this point will be guaranteed to be odd
\end{lstlisting}


\end{document}