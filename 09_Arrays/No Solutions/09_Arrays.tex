\documentclass{article}

\usepackage [margin=1in]{geometry}
\usepackage[usenames,dvipsnames]{color}
\usepackage{listings}
\lstset{language=Java,
	basicstyle=\ttfamily,
	keywordstyle=\color{blue}\ttfamily,
	stringstyle=\color{red}\ttfamily,
	showstringspaces=false,
	commentstyle=\color{ForestGreen}\ttfamily,
	morecomment=[l][\color{magenta}]{\#}}
\usepackage [autostyle]{csquotes}
\usepackage [english]{babel}
\usepackage {verbatim}

\title{CSE110 Review Questions}
\author{Prepared by Ryan Dougherty}
\date{}

\begin{document}
\maketitle

\section{Arrays}

\newcounter{question_num}

\setcounter{question_num}{1}
\paragraph{Question \arabic{question_num}}
What are the indices for the first and last positions of any array?

\addtocounter{question_num}{1}
\paragraph{Question \arabic{question_num}}
Immediately after instantiating a new array of primitives (ints, doubles, etc.), what fills the array? What about an array of objects?

\addtocounter{question_num}{1}
\paragraph{Question \arabic{question_num}}
What happens when you try to access an array element past the end of the array?

\addtocounter{question_num}{1}
\paragraph{Question \arabic{question_num}}
Instantiate three arrays called x, y, and z of type int, String, and BackAccount (respectively), all of size 10.

\addtocounter{question_num}{1}
\paragraph{Question \arabic{question_num}}
Write a for-loop to double each element in an array x of type int.

\addtocounter{question_num}{1}
\paragraph{Question \arabic{question_num}}
Write code to store the largest number in an int array x into a variable called max.

\addtocounter{question_num}{1}
\paragraph{Question \arabic{question_num}}
Write code to count how many numbers in the array are strictly larger than 4, and store that total in a variable called total.

\addtocounter{question_num}{1}
\paragraph{Question \arabic{question_num}}
Write code to print out every other element in an array separated by tabs.

\addtocounter{question_num}{1}
\paragraph{Question \arabic{question_num}}
Write code to shift each number one place to the right (Note: there will be 2 copies of the 1st element when the code finishes).

\addtocounter{question_num}{1}
\paragraph{Question \arabic{question_num}}
Write code to print the contents of an array in reverse order, one element for each line.

\addtocounter{question_num}{1}
\paragraph{Question \arabic{question_num}}
Use the following array x to answer the following questions:
\begin{table}[h]
\begin{tabular}{lllllll}
4 & 8 & 5 & 1 & 6 & 3 & 2
\end{tabular}
\end{table}
\newline
a) What value is given by x[1]?
\newline b) What value is given by x[6]?
\newline c) What value is given by x[7]?
\newline d) What value is given by x.length?

\end{document}