\documentclass{article}

\usepackage [margin=1in]{geometry}
\usepackage[usenames,dvipsnames]{color}
\usepackage{listings}
\lstset{language=Java,
	basicstyle=\ttfamily,
	keywordstyle=\color{blue}\ttfamily,
	stringstyle=\color{red}\ttfamily,
	showstringspaces=false,
	commentstyle=\color{ForestGreen}\ttfamily,
	morecomment=[l][\color{magenta}]{\#}}
\usepackage [autostyle]{csquotes}
\usepackage [english]{babel}
\usepackage {verbatim}

\title{CSE110 Review Questions (Solutions)}
\author{Prepared by Ryan Dougherty}
\date{}

\begin{document}
\maketitle

\section{Arrays}

\newcounter{question_num}

\setcounter{question_num}{1}
\paragraph{Question \arabic{question_num}}
What are the indices for the first and last positions of any array?
\newline {\color{red}Answers: x[0], x[x.length - 1]}

\addtocounter{question_num}{1}
\paragraph{Question \arabic{question_num}}
Immediately after instantiating a new array of primitives (ints, doubles, etc.), what fills the array? What about an array of objects?
\newline {\color{red}Answers: 0, null}

\addtocounter{question_num}{1}
\paragraph{Question \arabic{question_num}}
What happens when you try to access an array element past the end of the array?
\newline {\color{red}Answer: The code throws an ArrayIndexOutOfBoundsException.}

\addtocounter{question_num}{1}
\paragraph{Question \arabic{question_num}}
Instantiate three arrays called x, y, and z of type int, String, and BackAccount (respectively), all of size 10.
\newline {\color{red}Answers:}
\begin{lstlisting}
int[] x = new int[10];
String[] y = new String[10];
BankAccount[] z = new BankAccount[10];
\end{lstlisting}

\addtocounter{question_num}{1}
\paragraph{Question \arabic{question_num}}
Write a for-loop to double each element in an array x of type int.
\newline {\color{red}Answer:}
\begin{lstlisting}
for (int i = 0; i < x.length; i++)
	x[i] *= 2;
\end{lstlisting}

\addtocounter{question_num}{1}
\paragraph{Question \arabic{question_num}}
Write code to store the largest number in an int array x into a variable called max.
\newline {\color{red}Answer:}
\begin{lstlisting}
int max = x[0];
for (int i = 1; i < x.length; i++)
	if (x[i] > max)
		max = num[i];
\end{lstlisting}


\addtocounter{question_num}{1}
\paragraph{Question \arabic{question_num}}
Write code to count how many numbers in the array are strictly larger than 4, and store that total in a variable called total.
\newline {\color{red}Answer:}
\begin{lstlisting}
int total = 0;
for (int i = 0; i < x.length; i++)
	if (x[i] > 4)
		total++;
\end{lstlisting}

\addtocounter{question_num}{1}
\paragraph{Question \arabic{question_num}}
Write code to print out every other element in an array separated by tabs.
\newline {\color{red}Answer:}
\begin{lstlisting}
for (int i = 0; i < x.length; i+=2)
	System.out.println(x[i] + "\t");
\end{lstlisting}

\addtocounter{question_num}{1}
\paragraph{Question \arabic{question_num}}
Write code to shift each number one place to the right (Note: there will be 2 copies of the 1st element when the code finishes).
\newline {\color{red}Answer:}
\begin{lstlisting}
for (int i = x.length-2; i >= 0; i--)
	x[i+1] = x[i];
\end{lstlisting}

\addtocounter{question_num}{1}
\paragraph{Question \arabic{question_num}}
Write code to print the contents of an array in reverse order, one element for each line.
\newline {\color{red}Answer:}
\begin{lstlisting}
for (int i = x.length-1; i >= 0; i--)
	System.out.println(x[i]);
\end{lstlisting}

\addtocounter{question_num}{1}
\paragraph{Question \arabic{question_num}}
Use the following array x to answer the following questions:
\begin{table}[h]
\begin{tabular}{lllllll}
4 & 8 & 5 & 1 & 6 & 3 & 2
\end{tabular}
\end{table}
\newline
a) What value is given by x[1]?
\newline b) What value is given by x[6]?
\newline c) What value is given by x[7]?
\newline d) What value is given by x.length?
\newline {\color{red}Answers:
\newline a) 8
\newline b) 2
\newline c) ArrayIndexOutOfBoundsException thrown
\newline d) 7
}
\end{document}