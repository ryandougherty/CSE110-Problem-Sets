\documentclass{article}

\usepackage [margin=1in]{geometry}
\usepackage[usenames,dvipsnames]{color}
\usepackage{listings}
\lstset{language=Java,
	basicstyle=\ttfamily,
	keywordstyle=\color{blue}\ttfamily,
	stringstyle=\color{red}\ttfamily,
	showstringspaces=false,
	commentstyle=\color{ForestGreen}\ttfamily,
	morecomment=[l][\color{magenta}]{\#}}
\usepackage [autostyle]{csquotes}
\usepackage [english]{babel}
\usepackage {verbatim}


\date{}

\begin{document}
\title{\textbf{CSE110 Review Questions \\
Prepared by Ryan Dougherty}}
\maketitle

\section*{Introduction to Classes}

\newcounter{question_num}

\setcounter{question_num}{1}
\paragraph{Question \arabic{question_num}}
Which of the following enforces Encapsulation?
\newline a) Make instance variables private
\newline b) Make methods public
\newline c) Make the class final
\newline d) Both a and b
\newline e) All of the above

\addtocounter{question_num}{1}
\paragraph{Question \arabic{question_num}}
Use the following class to answer the questions below:
\begin{lstlisting}
public class Store {
	private int quantity;
	private double price;

	public Store(int q, double p) {
		quantity = q;
		price = p;
	}

	public int getQuantity() {
		return quantity;
	}

	public void setPrice(double p) {
		price = p;
	}

	public double calcTotal() {
		return price * quantity;
	}
}
\end{lstlisting}
a) What is the name of the class?
\newline b) List all instance variables of the class.
\newline c) List all methods of the class.
\newline d) List all mutators in the class.
\newline e) List all accessors in the class.
\newline f) List which method is the constructor.
\newline g) Write a mutator for the quantity.
\newline h) Write an accessor for the price.
\newline i) Write a line of code that will create an instance called videoStore that has quantity 100 and a price of 5.99.
\newline j) Call the calcTotal method with the videoStore object (from part i) to print out the total.

\addtocounter{question_num}{1}
\paragraph{Question \arabic{question_num}}
True or False? If no constructor is provided, then Java automatically provides a default constructor.

\addtocounter{question_num}{1}
\paragraph{Question \arabic{question_num}}
True or False? A method must have at least 1 return statement.

\addtocounter{question_num}{1}
\paragraph{Question \arabic{question_num}}
Correct the following class definition if you think it will not work:
\begin{lstlisting}
public class Student {
	private String name, major;

	public Student() {
		name = "???";
		major = "xxx";
	}

	public Student(String n, String m) {
		n = name;
		m = major;
	}

	public String getMajor() {
		return m;
	}

	public String getName() {
		return n;
	}
}
\end{lstlisting}

\addtocounter{question_num}{1}
\paragraph{Question \arabic{question_num}}
Implement a class called AsuStudent. The class should keep track of the student's name, number of classes registered, hours spent per week for a class (consider a student devotes the same amount of time for each of his/her classes per week). Implement a toString method to show the name and number of classes registered by a student, a getName method to return the name of the student, a getTotalHours method to return the total number of hours per week, and a setHours method to set the number of hours the student devotes for each class.

\end{document}