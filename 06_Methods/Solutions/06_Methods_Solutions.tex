\documentclass{article}

\usepackage [margin=1in]{geometry}
\usepackage[usenames,dvipsnames]{color}
\usepackage{listings}
\lstset{language=Java,
	basicstyle=\ttfamily,
	keywordstyle=\color{blue}\ttfamily,
	stringstyle=\color{red}\ttfamily,
	showstringspaces=false,
	commentstyle=\color{ForestGreen}\ttfamily,
	morecomment=[l][\color{magenta}]{\#}}
\usepackage [autostyle]{csquotes}
\usepackage [english]{babel}
\usepackage {verbatim}

\title{CSE110 Review Questions (Solutions)}
\author{Prepared by Ryan Dougherty}
\date{}

\begin{document}
\maketitle

\section*{Methods}

\newcounter{question_num}

\setcounter{question_num}{1}
\paragraph{Question \arabic{question_num}}
Write a boolean method called allDifferent that takes 3 int numbers and returns true if the numbers are all different and false otherwise.
\newline {\color{red} Answer:}
\begin{lstlisting}
public boolean allDifferent(int num1, int num2, int num3) {
	if (num1 != num2 && num1 != num3 && num2 != num3) {
		return true;
	} else {
		return false;
	}
}
\end{lstlisting}

\addtocounter{question_num}{1}
\paragraph{Question \arabic{question_num}}
Write a boolean method called isPrime that takes in an int number, and returns true if the number is prime, and false otherwise.
\newline {\color{red}Answer:}
\begin{lstlisting}
public boolean isPrime(int n) {
	// Question: how can we improve the performance of this loop? 
	// (Hint: what is the max number relative to n that can divide into n?)
	for (int i = 2; i < n; i++) {
		if (n % i == 0) {
			return false;
		}
	}
	return true;
}
\end{lstlisting}

\addtocounter{question_num}{1}
\paragraph{Question \arabic{question_num}}
Write the output generated by the following program:
\begin{lstlisting}
public class Two {
	private double real, imag;

	public Two(double initReal, double initImag) {
		real = initReal;
		imag = initImag;
	}

	public double getReal() {
		return real;
	}

	public double getImag() {
		return imag;
	}

	public Two mystery(Two rhs) {
		Two temp = new Two(getReal() + rhs.getReal(), getImag() + rhs.getImag());
		return temp;
	}
}

public class Test {
	public static void main(String[] args) {
		Two a = new Two(1.2, 3.4);
		Two b = a.mystery(a);
		Two c = b.mystery(b);

		System.out.println("1. " + a.getReal());
		System.out.println("2. " + a.getImag());
		System.out.println("3. " + b.getReal());
		System.out.println("4. " + b.getImag());
		System.out.println("5. " + c.getImag());
	}
}
\end{lstlisting}
{\color{red}Answers:
\newline 1. 1.2
\newline 2. 3.4
\newline 3. 2.4
\newline 4. 6.8
\newline 5. 10.2}

\addtocounter{question_num}{1}
\paragraph{Question \arabic{question_num}}
Using these 2 classes, write the output of the following program:
\begin{lstlisting}
public class CDPlayer {
	private int totalTime;

	public CDPlayer() {
		totalTime = 0;
	}

	public int totalPlayTime() {
		return totalTime;
	}

	public void play(CDTrack aTrack) {
		totalTime += aTrack.getPlayTime();
	}
}

public class CDTrack {
	private String myTitle;
	private int myPlayTime, myTimesPlayed;

	public CDTrack(String trackTitle, int playTime) {
		myTitle = trackTitle;
		myPlayTime = playTime;
		myTimesPlayed = 0;
	}

	public int getPlayTime() {
		return myPlayTime;
	}

	public void wasPlayed() {
		myTimesPlayed++;
	}

	public String toString() {
		String result = "";
		int minutes = myPlayTime / 60;
		int seconds = myPlayTime % 60;
		result += myTitle + " " + minutes + ":" + seconds;
		result += " #plays = " + myTimesPlayed;
		return result;
	}
}

public class RunCDPlayer {
	public static void main(String[] args) {
		CDTrack t1 = new CDTrack("Day Tripper", 150);
		CDTrack t2 = new CDTrack("We Can Work it Out", 200);
		CDTrack t3 = new CDTrack("Paperback Writer", 138);

		CDPlayer diskPlayer = new CDPlayer();
		t1.wasPlayed();
		diskPlayer.play(t1);
		t2.wasPlayed();
		diskPlayer.play(t2);
		t1.wasPlayed();
		diskPlayer.play(t1);

		System.out.println(t1.toString());
		System.out.println(t2.toString());
		System.out.println(t3.toString());
		System.out.println("Totak play time: " + (diskPlayer.totalPlayTime() / 60) + ":" + (diskPlayer.totalPlayTime() % 60));
	}
}
\end{lstlisting}
{\color{red}Answers:
\newline Day Tripper 2:30 \#plays = 2
\newline We Can Work it Out 3:20 \#plays = 1
\newline Paperback Writer 2:18 \#plays = 0
\newline Total play time: 8:20}

\end{document}