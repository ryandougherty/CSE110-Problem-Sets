\documentclass{article}

\usepackage [margin=1in]{geometry}
\usepackage[usenames,dvipsnames]{color}
\usepackage{listings}
\lstset{language=Java,
	basicstyle=\ttfamily,
	keywordstyle=\color{blue}\ttfamily,
	stringstyle=\color{red}\ttfamily,
	showstringspaces=false,
	commentstyle=\color{ForestGreen}\ttfamily,
	morecomment=[l][\color{magenta}]{\#}}
\usepackage [autostyle]{csquotes}
\usepackage [english]{babel}
\usepackage {verbatim}

\title{CSE110 Review Questions}
\author{Prepared by Ryan Dougherty}
\date{}

\begin{document}
\maketitle

\newcounter{question_num}
\setcounter{question_num}{1}

\section{Data Types}

\setcounter{question_num}{1}
\paragraph{Question \arabic{question_num}}
Give the output of the following program:
\begin{lstlisting}
public class Example {
	public static void main(String[] args) {
		int y = 2, z = 1;
		z = y * 2;
		System.out.print(y + z);
	}
}
\end{lstlisting}

\addtocounter{question_num}{1}
\paragraph{Question \arabic{question_num}}
Consider the following program:
\begin{lstlisting}
public class Example {
	public static void main(String[] args) {
		String str = new String("Arizona state university");
		char ch1 = str.toLowerCase().toUpperCase().charAt(0);
		char ch2 = str.toUpperCase().charAt(8);
		char ch3 = str.toUpperCase().charAt(str.length() - 1);
		System.out.println("character 1 is: " + ch1);
		System.out.println("character 2 is: " + ch2);
		System.out.println("character 3 is: " + ch3);
	}
}
\end{lstlisting}
What will be the output?

\addtocounter{question_num}{1}
\paragraph{Question \arabic{question_num}}
Consider the following program:
\begin{lstlisting}
public class Example {
	public static void main(String[] args) {
		int num1 = 4, num2 = 5;
		System.out.println("4" + "5");
		System.out.println(num1 + num2);
		System.out.println("num1" + "num2");
		System.out.println(4+5);
	}
}
\end{lstlisting}
What will be the output?

\addtocounter{question_num}{1}
\paragraph{Question \arabic{question_num}}
Which of the following invokes the method length() of the object str and stores the result in val of type int?
\begin{lstlisting}
a) int val = str.length();
b) int val = length.str();
c) int val = length().str;
d) int val = length(str);
\end{lstlisting}

\addtocounter{question_num}{1}
\paragraph{Question \arabic{question_num}}
Evaluate each of the following expressions.
\begin{lstlisting}
String s = "Programming is Fun";
String t = "Workshop is cool";
a) System.out.println(s.charAt(0) + t.substring(3, 4));
b) System.out.println(t.substring(7));
\end{lstlisting}

\addtocounter{question_num}{1}
\paragraph{Question \arabic{question_num}}
Evaluate each of the following expressions.
\begin{lstlisting}
int j = 11;
int k = 3;
String s = "Ford Rivers";
a) j / k
b) j \% k
c) s.substring(1, 5)
d) s.length()
e) s.charAt(3)
\end{lstlisting}

\addtocounter{question_num}{1}
\paragraph{Question \arabic{question_num}}
True or False? The type char is a primitive data type.

\addtocounter{question_num}{1}
\paragraph{Question \arabic{question_num}}
True or False? The type String is a primitive data type.

\addtocounter{question_num}{1}
\paragraph{Question \arabic{question_num}}
Write a Java program that asks the user for the radius of a circle and finds the area of the circle.

\addtocounter{question_num}{1}
\paragraph{Question \arabic{question_num}}
Write a Java program that prompts the user to enter 2 integers. Print the smaller of the 2 integers.

\end{document}